\documentclass{scrartcl}
\usepackage{riley}
\usepackage{riley-libertine}

\usepackage{tikz}
\usetikzlibrary{calc}

\newcommand{\black}[1]{{\color{black}{#1}}}
\DeclareMathOperator{\sech}{sech}
\DeclareMathOperator{\csch}{csch}
\DeclareMathOperator{\SO}{SO}
\begin{document}
%\color{darkgray}
Complex\hfil Hyperbolic
\begin{align*}
  i^2 &\define {-1} &  j^2 &\define {+1} \\
  i(x+iy) &= -y + ix & j(x+jy) &=y +jx
  \intertext{
  \begin{tikzpicture}
    \draw[gray](-2,0) -- (2,0) (0,-2) -- (0,2);
    \draw[->](0,0) -- (1,2)  coordinate [label=right:{\(z\)}];
    \draw[->](0,0) -- (-2,1) coordinate [label=above:{\(iz\)}];
  \end{tikzpicture}
  \hfill
  \begin{tikzpicture}
    \draw[gray,dashed](-2,-2) -- (2,2);
    \draw[gray](-2,0) -- (2,0) (0,-2) -- (0,2);
    \draw[->](0,0) -- (1,2) coordinate [label=right:{\(z\)}];
    \draw[->](0,0) -- (2,1) coordinate [label=right:{\(jz\)}];
  \end{tikzpicture}
  }
  \intertext{Conjugation and norm}
  (x+iy)^* &= x-iy & (x+jy)^* &= x-jy
  \intertext{
  \begin{tikzpicture}
    \draw[gray](-2,0) -- (2,0) (0,-2) -- (0,2);
    \draw[->](0,0) -- (1,2)  coordinate [label=right:{\(z\)}];
    \draw[->](0,0) -- (1,-2) coordinate [label=right:{\(z^*\)}];
  \end{tikzpicture}
  \hfill
  \begin{tikzpicture}
    \draw[gray](-2,0) -- (2,0) (0,-2) -- (0,2);
    \draw[->](0,0) -- (1,2) coordinate [label=right:{\(z\)}];
    \draw[->](0,0) -- (1,-2) coordinate [label=right:{\(z^*\)}];
  \end{tikzpicture}
  }
  \abs{z}^2&\define zz^*=(x+iy)(x-iy)  & \abs{z}^2 &\define zz^*=(x+jy)(x-jy) \\
      &= x^2 +y^2 & &=x^2-y^2
\intertext{
  \begin{tikzpicture}
    \draw[gray] (-2,0) -- (2,0) (0,-2) -- (0,2);
    \draw (0,0) circle (1);
  \end{tikzpicture}
                      \hfill
  \begin{tikzpicture}
    \clip (-2,-2) rectangle (2,2);
    \draw[gray](-2,0) -- (2,0) (0,-2) -- (0,2);
    \draw[gray,dashed](-2,-2) -- (2,2) (-2,2) -- (2,-2);
    \draw[samples=5, domain=-2:2,smooth,variable=\t]
    plot ({cosh \t}, {sinh \t})
    plot (-{cosh \t}, {sinh \t})
    ;
  \end{tikzpicture}
                      }
  \intertext{Exponentiation}
  i^n&=
       \begin{cases}
         1  &\color{gray} n \equiv0 \pmod 4\\
         i  &\color{gray} n \equiv1\pmod 4\\
         -1 &\color{gray} n \equiv2\pmod 4\\
         -i &\color{gray} n \equiv3\pmod 4
       \end{cases}
       & j^n&=
              \begin{cases}
                1&\color{gray} n \equiv0 \pmod 4\\
                j&\color{gray} n \equiv1\pmod 4\\
                1&\color{gray} n \equiv2\pmod 4\\
                j&\color{gray} n \equiv3\pmod 4
              \end{cases} \\
  e^{it} &= 1(\dots) + i(\dots) & e^{jt} &= 1(\dots) + j(\dots)  \\
        & -1(\dots)-i(\dots) && +1(\dots) +j(\dots) \\
  \intertext{
    \begin{tikzpicture}[scale=3, every node/.style={color=darkgray}]
      \clip (-1,-1.4) rectangle (1.6,1.4);
      \draw[gray] (0,0) circle (1);
      \draw let \n0 = {1}
      in (0,0) -- node[below] {\(1\)}
      ++ (1,0) -- node[right]{\(it\)}
      ++(0,{\n0}) -- node[above]{\(-t^2/2\)}
      ++ ({-\n0^2/2},0) -- node[below left]{\(-it^3/6\)}
      ++(0,{-\n0^3/6}) --
      ++ ({\n0^4/24,0}) --
      ++ (0,{\n0^5/120});
    \end{tikzpicture}
    \hfill
    \begin{tikzpicture}[scale=3, every node/.style={color=darkgray}]
      \clip (0,-1.4) rectangle (2,1.4);
      \draw[gray,samples=5, domain=-2:2,smooth,variable=\t]
      plot ({cosh \t}, {sinh \t});
      \draw let \n0 = {1}
      in (0,0) -- node[below]{\(1\)}
      ++ (1,0) -- node[left]{\(jt\)}
      ++(0,{\n0}) -- node[below]{\(t^2/2\)}
      ++ ({\n0^2/2},0) -- node[right]{\(jt^3/6\)}
      ++(0,{\n0^3/6}) --
      ++ ({\n0^4/24,0}) --
      ++ (0,{\n0^5/120});
    \end{tikzpicture}
  }
  e^{it}&= \cos t + i \sin t & e^{jt} &=\cosh t + j \sinh t
\intertext{
  \begin{tikzpicture}[scale=3]
    \draw[gray] (-1,0) -- (1,0) (0,-1) -- (0,1);
    \draw[gray] (0,0) circle (1);
    \def\t{1}
    \def\x{{cos(\t r)}}
    \def\y{{sin(\t r)}}
    \draw[->] (0,0) --  (\x,\y) coordinate[label=above right:{\(e^{it}\)}];
    \draw (0,0)--node[below]{\(\cos t\)} (\x,0)-- node[right]{\(i\sin t\)}(\x,\y);
  \end{tikzpicture}
  \hfill
  \begin{tikzpicture}[scale=3]
    \clip (0,-1) rectangle (1.8,1);
    \draw[gray] (-2,0) -- (4,0) (0,-2) -- (0,2);
    \draw[gray,dashed](0,0) -- (2,2) (0,0) -- (2,-2);
    \draw[gray,samples=5, domain=-2:2,smooth,variable=\t]
    plot ({cosh \t}, {sinh \t})
    ;
    \def\t{.7}
    \def\x{{cosh \t}}
    \def\y{{sinh \t}}
    \draw[->] (0,0) -- (\x,\y) coordinate[label=above left:{\(e^{jt}\)}];
    \draw (0,0)--node[below]{\(\cosh t\)}(\x,0) --node[right]{\(j \sinh t\)} (\x,\y);
  \end{tikzpicture}
                      }
  \intertext{Pythagorean identity}
  e^{-it}&= \cos (-t) + i \sin (-t) & e^{-jt} &=\cosh (-t) + j \sinh (-t) \\
      &= \cos t - i \sin t  &      &=\cosh t - j \sinh t \\
  e^{-it} &= \paren{e^{it}}^* & e^{-jt} &= \paren{e^{jt}}^* \\
  \abs{e^{it}}^2 &= \paren{e^{it}}\paren{e^{it}}^* & \abs{e^{jt}}^2 &= \paren{e^{jt}}\paren{e^{jt}}^* \\
     &= e^{it}e^{-it} &  & =e^{jt}e^{-jt} \\
      &=1  & &=1 \\
  1&=\cos^2 t + \sin^2 t & 1&= \cosh^2t -\sinh^2t
  \intertext{Pythagorean corollaries}
  \cos^2 t &= 1-\sin^2t & \cosh^2 t &= 1+\sinh^2 t \\
  \sin^2 t &= -\cos^2t+1 & \sinh^2 t &= \cosh^2t -1 \\
  \sec^2 t &= 1 + \tan^2 t & \sech^2 t&= 1 -\tanh^2 t \\
  \csc^2 t &= \cot^2 t+1 & \csch^2 t&= \coth^2 t -1
  \intertext{Angle sum formulae}
  e^{i(A+B)} &= e^{iA}e^{iB} & e^{j(A+B)} &= e^{jA}e^{jB} \\
   &= \paren{c_A + i s_A}\paren{c_B + is_B}
              &&= \paren{c_A + j s_A}\paren{c_B + js_B}\\
  &
    \begin{array}{c|cc}
          &c_B & is_B    \\
      \hline
      c_A &c_Ac_B&ic_As_B \\
      is_A&ic_Bs_A& -s_As_B
    \end{array}
    &&
    \begin{array}{c|cc}
          &c_B & js_B    \\
      \hline
      c_A &c_Ac_B&jc_As_B \\
      js_A&jc_Bs_A& +s_As_B
    \end{array} \\
  \begin{matrix}
    \phantom{+}\cos(A+B) \\
    +i\sin(A+B)
  \end{matrix}
  &=
    \begin{matrix}
      \phantom{+} c_Ac_B -s_As_B \\
      +i\paren{c_Bs_A +c_As_B}
    \end{matrix}
  & \begin{matrix}
    \phantom{+}\cosh(A+B) \\
    +j\sinh(A+B)
  \end{matrix}
  &=
    \begin{matrix}
      \phantom{+}c_Ac_B +s_As_B \\
      +j\paren{c_Bs_A +c_As_B}
    \end{matrix}
  \intertext{Double angle}
    e^{2it}&= \paren{e^{it}}^2 & e^{2jt}&=\paren{e^{jt}}^2  \\
  \cos 2t &= \cos^2 t- \sin^2 t & \cosh 2t &= \cosh^2t +\sinh^2 t  \\
  i\sin 2t &= i\,2\cos t\sin t& j\sinh 2t &= j\,2 \cosh t \sinh t
  \intertext{Double angle + Pythagorean}
  \cos 2t &= \cos^2t - (1-\cos^2t)&\cosh 2t &= \cosh^2t +(\cosh^2t - 1) \\
          &= 2\cos^2 t -1         &         &= 2\cosh^2t -1 \\
          &= (1-\sin^2t ) - \sin^2t &       &= (1+\sinh^2 t ) + \sinh^2 t \\
          &=1 -2\sin^2t           &         &= 1 + 2\sinh^2 t
  \intertext{Derivatives}
    f(t)&\define e^{it} & f(t)&\define e^{jt} \\
  f'(t) &= ie^{it} & f'(t)&= je^{jt} \\
  &=i\paren{\cos t+ i\sin t} &&=j\paren{\cosh t + j\sinh t} \\
  \cos' t +i\sin' t &= -\sin t +i\cos  t & \cosh' t + j\sinh' t&= \sinh t + j\cosh t
\intertext{
  \begin{tikzpicture}
    \draw[gray] (-2,0) -- (2,0) (0,-2) -- (0,2);
    \draw (0,0) circle (1);
    \def\t{1}
    \draw[->] (0,0) -- ({cos( \t r)}, {sin( \t r)});
    \draw[->] ({cos( \t r)}, {sin( \t r)}) -- ({cos( \t r) - sin(\t r)}, {sin( \t r) + cos(\t r)}) ;
  \end{tikzpicture}
  \hfill
  \begin{tikzpicture}
    \clip (-2,-2) rectangle (2,2);
    \draw[gray] (-2,0) -- (2,0) (0,-2) -- (0,2);
    \draw[gray,dashed](0,0) -- (2,2) (0,0) -- (2,-2);
    \draw[samples=5, domain=-2:2,smooth,variable=\t]
    plot ({cosh \t}, {sinh \t})
    ;
    \def\t{.4}
    \draw[->] (0,0) -- ({cosh( \t)}, {sinh( \t)});
    \draw[->] ({cosh(\t}, {sinh(\t)}) -- ({cosh(\t) + sinh(\t)}, {sinh(\t) + cosh(\t)}) ;
  \end{tikzpicture}
  }
\intertext{
  \begin{tikzpicture}
    \draw[gray] (-2,0) -- (2,0) (0,-2) -- (0,2);
    \draw (0,0) circle (1);
    \def\t{1}
    \draw[->] (0,0) -- ({cos( \t r)}, {sin( \t r)});
    \draw[->,lightgray] ({cos( \t r)}, {sin( \t r)}) -- ({cos( \t r) - sin(\t r)}, {sin( \t r) + cos(\t r)}) ;
    \draw[->] (0,0) -- ({- sin(\t r)}, { cos(\t r)}) ;
  \end{tikzpicture}
  \hfill
  \begin{tikzpicture}
    \clip (-2,-2) rectangle (2,2);
    \draw[gray] (-2,0) -- (2,0) (0,-2) -- (0,2);
    \draw[gray,dashed](0,0) -- (2,2) (0,0) -- (2,-2);
    \draw[samples=5, domain=-2:2,smooth,variable=\t]
    plot ({cosh \t}, {sinh \t})
    plot ({sinh \t}, {cosh \t})
    ;
    \def\t{.4}
    \draw[->] (0,0) -- ({cosh( \t)}, {sinh( \t)});
    \draw[lightgray,->] ({cosh(\t}, {sinh(\t)}) -- ({cosh(\t) + sinh(\t)}, {sinh(\t) + cosh(\t)}) ;
    \draw[->] (0,0) -- ({sinh(\t)}, {cosh(\t)}) ;
  \end{tikzpicture}
                      }
  \intertext{Dot and cross product}
  u^*v &= \paren{\alpha - i\beta}\paren{\gamma + i\delta} & u^*v &= \paren{\alpha - j\beta}\paren{\gamma + j\delta} \\
  &
    \begin{array}{c|cc}
            &\gamma &i\delta \\
      \hline
      \alpha& \alpha\gamma  & i\alpha\delta \\
      -i\beta& -i\beta\gamma& + \beta\delta
    \end{array}
    &&
    \begin{array}{c|cc}
            &\gamma &i\delta \\
      \hline
      \alpha& \alpha\gamma  & +j\alpha\delta \\
      -j\beta& -j\beta\gamma& - \beta\delta
    \end{array} \\
      &= (\alpha\gamma + \beta\delta) + i (\alpha\delta - \beta\gamma)
      &&= (\alpha\gamma - \beta\delta) + j (\alpha\delta - \beta\gamma) \\
     u^*v &= (u\cdot v) + i(u\times v) & u^*v &=(u\cdot v) +j (u\times v) \\
  (iz)\cdot z &= \Re (iz^*z)= \Re(i\abs{z}^2)=0  & (jz)\cdot z &= \Re (j z^*z)=\Re(j\abs{z}^2)=0
\end{align*}
\begin{align*}
  \intertext{Arc length}
   f(t) &\define e^{it} & f(t)&\define e^{jt} \\
  S &= \int \sqrt{\abs{f'(t)}^2} dt& S& = \int \sqrt{-\abs{f'(t)}^2}dt \\
                 &= \int \sqrt{\abs{ie^{it}}^2} dt = \int\sqrt{1}dt & & = \int \sqrt{-\abs{je^{jt}}^2} dt=\int\sqrt{-(-1)}dt\\
                 &= \Delta t & &=\Delta t
  \intertext{
    \begin{tikzpicture}[scale=2, baseline={(0,0)}]
      \draw[gray] (0,0)  circle (1);
      \draw let \n0 = 0, \n1 = 1, \n Δ ={\n1-\n0}
      in
      (0,0) -- ({cos(\n0 r)},{sin(\n0 r)})
      arc[radius=1,start angle=\n0 r , delta angle = \n Δ r]
      node[midway, right]{\(\Delta t\)}
      -- cycle
      ;
    \end{tikzpicture}
    \hfill
    \begin{tikzpicture}[scale=2,baseline={(0,0)}]
      \clip (0,-1.3) rectangle (2,1.3);
      \draw[gray,samples=5, domain=-2:2,smooth,variable=\t]
      plot ({cosh \t}, {sinh \t});
%
      \draw let \n0 = 0, \n1 = 1
      in
      (0,0) -- ({cosh(\n0 )},{sinh(\n0)})
      ({cosh(\n1)}, {sinh(\n1)})
      -- (0,0)
      [samples =5, domain=\n0:\n1,smooth,variable=\t] plot ({cosh \t}, {sinh \t})
      node at (1.3,.5) {\(\Delta t\)}
      ;
    \end{tikzpicture}
  }
  \intertext{Sector area}
  \intertext{
                                 \hfil
    \begin{tikzpicture}[scale=4]
      \draw[lightgray,fill=lightgray!20] (0,0) -- (1,0) -- (.94,.2) -- cycle;
      \draw[lightgray] (0,0) --(1,0)--(.94,.2) -- (-.06,.2) -- cycle;
      \clip (0,-.2)  rectangle (1.4,.4);
      \draw[gray] (.02,-.2) circle (1);
      \draw (0,0) --  node[below]{\(f(t)\)} (1,0)
      (1,0) --  node[right]{\(f'(t)\)} (.94,.2)
      ;
    \end{tikzpicture}
  }
  A &= \frac{1}{2} \int { f(t)\times f'(t)}dt& A&=\frac12 \int{f(t)\times f'(t)} dt \\
  &=\frac12 \int \Im\paren[\Big]{f^*(t)f'(t)}dt &&=\frac12 \int \Im\paren{f^*(t)f'(t)} dt \\
                 &=\frac12 \int \Im\paren{e^{-it}ie^{it}}dt &&=\frac12 \int \Im\paren{e^{-jt}je^{jt}} dt \\
                 &=\frac12\int 1\,dt &&=\frac12\int 1\,dt \\
  &= \frac{\Delta t}2 & &=\frac{\Delta t}2
  \intertext{
    \begin{tikzpicture}[scale=2, baseline={(0,0)}]
      \draw[gray] (0,0)  circle (1);
      \draw[fill=lightgray!20] let \n0 = 0, \n1 = 1, \n Δ ={\n1-\n0}
      in
      (0,0) -- ({cos(\n0 r)},{sin(\n0 r)})
      arc[radius=1,start angle=\n0 r , delta angle = \n Δ r]
      -- cycle
      ;
    \end{tikzpicture}
    \hfill
    \begin{tikzpicture}[scale=2,baseline={(0,0)}]
      \clip (0,-1.3) rectangle (2,1.3);
      \draw[gray,samples=8, domain=-1.2:1.2,smooth,variable=\t]
      plot ({cosh \t}, {sinh \t});
%
      \begin{scope}
        % I could not get this clipping right. This seems to work but I don't know why
        \clip
        [samples=8, domain=-1:0,smooth,variable=\t]
 plot ({cosh (-\t)}, {sinh (-\t)})
         (-2,-2) rectangle (2,2)
         ;
        \path[fill=lightgray!20] let \n0 = 0, \n1 = 1
        in
        (0,0) -- ({cosh(\n0 )},{sinh(\n0)}) --
        ({cosh(\n1)}, {sinh(\n1)})
        -- cycle
        ;
      \end{scope}
      %
      \draw let \n0 = 0, \n1 = 1
      in
      (0,0) -- ({cosh(\n0 )},{sinh(\n0)})
      ({cosh(\n1)}, {sinh(\n1)})
      -- (0,0)
      [samples =8, domain=\n0:\n1,smooth,variable=\t] plot ({cosh \t}, {sinh \t})
      ;
    \end{tikzpicture}
  }
 \intertext{\(\SO(2),\SO^+(+,-)\): Multiplying by imaginary exponents preserves inner product and area}
 \paren{e^{it}u}\cdot\paren{e^{it}v } && \paren{e^{jt}u}\cdot\paren{e^{jt}v} \\
 +i\paren{e^{it}u}\times\paren{e^{it}v } &= \paren{e^{it}u}^*\paren{e^{it}v} & +j\paren{e^{jt}u}\times\paren{e^{jt}v}&=\paren{e^{jt}u}^*\paren{e^{jt}v} \\
                 &= e^{-it}u^*e^{it}v & &=e^{-jt}u^*e^{jt}v \\
                 &=u^*v & & = u^* v \\
  &=u\cdot v + i\, u\times v & &=u\cdot v + j\, u\times v \\
  \begin{tikzpicture}[scale=.6,baseline=0]
    \draw[lightgray] (0,0) circle (1)
                     (0,0) circle (1.4) ;
    \draw (0,0) -- (1,0) --(1,1) --(0,1) --cycle;
  \end{tikzpicture}
  &\implies
  \begin{tikzpicture}[scale=.6,baseline=0]
    \draw[lightgray] (0,0) circle (1)
                     (0,0) circle (1.4) ;
    \draw (0,0) -- (.7,.7) --(0,1.4) --(-.7,.7) --cycle;
  \end{tikzpicture}
    &
  \begin{tikzpicture}[scale=.6]
    \clip (0,0) rectangle (2,2);
    \draw[lightgray,samples=5, domain=-2:2,smooth,variable=\t]
      plot ({cosh \t}, {sinh \t})
      plot ({sinh \t}, {cosh \t})
      (0,0) -- (2,2)
      ;
    \draw (0,0) -- (1,0) --(1,1) --(0,1) --cycle;
  \end{tikzpicture}
      &
      \implies
  \begin{tikzpicture}[scale=.6]
    \clip (0,-0.2) rectangle (4,4);
    \draw[lightgray,samples=5, domain=-2:2,smooth,variable=\t]
      plot ({cosh \t}, {sinh \t})
      plot ({sinh \t}, {cosh \t})
      (0,0) -- (4,4)
      ;
    \draw let \p1 = (1,0),
              \p2 = (0,1),
              \n0  = 1
      in
      (0,0) --
      ({cosh(\n0) * \x1 + sinh(\n0) * \y1}, {sinh(\n0) * \x1 + cosh(\n0) * \y1})
      -- +({cosh(\n0) * \x2 + sinh(\n0) * \y2}, {sinh(\n0) * \x2 + cosh(\n0) * \y2}) --
({cosh(\n0) * \x2 + sinh(\n0) * \y2}, {sinh(\n0) * \x2 + cosh(\n0) * \y2}) --cycle ;
      % ({cosh(\n1) * \x1 + \sinh(\n1) * \y1 },{sinh(\n1) * \x1 + cosh(\n1) * \y1 });
  \end{tikzpicture}
\end{align*}
\hfill
% tikz coordinates are stored in pt so eg \x1 will be way too big
\def\squishsquare#1#2#3{%
let \p A = (#1),
          \p B = (#2),
          \n0  = {#3},
          \p{tA}  = ($\X A *({cosh(\n0)},{sinh(\n0)})
                    +\Y A *({sinh (\n0)},{cosh(\n0)}) $),
          \p{tB}  = ($\X B *({cosh(\n0)},{sinh(\n0)})
                    +\Y B *({sinh (\n0)},{cosh(\n0)}) $)
 in
 (0,0) -- (\p{tA}) -- ($ (\p{tA}) + (\p{tB}) $) -- (\p{tB}) -- cycle
 }
\begin{tikzpicture}
  \clip (-2,-2) rectangle (2,2);
    \draw[lightgray,samples=5, domain=-2:2,smooth,variable=\t]
      plot ({cosh \t}, {sinh \t})
      plot ({sinh \t}, {cosh \t})
      plot ({-cosh \t}, {sinh \t})
      plot ({sinh \t}, {-cosh \t})
      (-4,-4) -- (4,4)
      (-4,4) -- (4,-4);
    \def\X#1{\x{#1}/1cm}
    \def\Y#1{\y{#1}/1cm}
    \def\U{1,0}
    \def\V{0,1}
    \draw[lightgray]
    \squishsquare{\U}{\V}{-.3}
    \squishsquare{\U}{\V}{-.6}
    \squishsquare{\U}{\V}{-.9}
    \squishsquare{\U}{\V}{-1.2}
    \squishsquare{\U}{\V}{.3}
    \squishsquare{\U}{\V}{.6}
    ;
    \draw
    \squishsquare{\U}{\V}{0};
\end{tikzpicture}
%
\begin{tikzpicture}
  \clip (-1,-2) rectangle (2,2);
    \draw[lightgray,samples=5, domain=-2:2,smooth,variable=\t]
      plot ({cosh \t}, {sinh \t})
      plot ({sinh \t}, {cosh \t})
      plot ({-cosh \t}, {sinh \t})
      plot ({sinh \t}, {-cosh \t})
      (-4,-4) -- (4,4)
      (-4,4) -- (4,-4);
    \def\X#1{\x{#1}/1cm}
    \def\Y#1{\y{#1}/1cm}
    \def\U{.5,.5}
    \def\V{.5,-.5}
    \draw[lightgray]
    \squishsquare{\U}{\V}{-.3}
    \squishsquare{\U}{\V}{-.6}
    \squishsquare{\U}{\V}{-.9}
    \squishsquare{\U}{\V}{-1.2}
    \squishsquare{\U}{\V}{.3}
    \squishsquare{\U}{\V}{.6}
    \squishsquare{\U}{\V}{.9}
    \squishsquare{\U}{\V}{1.2}
    ;
    \draw
    \squishsquare{\U}{\V}{0};
\end{tikzpicture}
%
\begin{tikzpicture}
  \clip (0,-2) rectangle (4,2);
    \draw[lightgray,samples=5, domain=-2:2,smooth,variable=\t]
      plot ({cosh \t}, {sinh \t})
      plot ({sinh \t}, {cosh \t})
      plot ({-cosh \t}, {sinh \t})
      plot ({sinh \t}, {-cosh \t})
      (-4,-4) -- (4,4)
      (-4,4) -- (4,-4);
    \def\X#1{\x{#1}/1cm}
    \def\Y#1{\y{#1}/1cm}
    \def\U{1.05,.3}
    \def\V{1.05,-.3}
    \draw[lightgray]
    \squishsquare{\U}{\V}{-.3}
    \squishsquare{\U}{\V}{-.6}
    \squishsquare{\U}{\V}{-.9}
    \squishsquare{\U}{\V}{-1.2}
    \squishsquare{\U}{\V}{.3}
    \squishsquare{\U}{\V}{.6}
    \squishsquare{\U}{\V}{.9}
    \squishsquare{\U}{\V}{1.2}
    ;
    \draw
    \squishsquare{\U}{\V}{0};
\end{tikzpicture}
\end{document}
