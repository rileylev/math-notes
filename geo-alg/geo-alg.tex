\documentclass{scrartcl}
\usepackage{amsmath}
\usepackage{riley}
\usepackage{riley-libertine}
\usepackage{xfrac}
\usepackage{multicol}
\usepackage[top=.9in,left=.9in,right=.9in]{geometry}
\begin{document}
\renewcommand{\b}{u}
\renewcommand{\R}{R}
\begin{multicols*}{2}
  \section{Rotation matrix facts}
  Start small
  \[
    R_\theta \define
    \begin{bmatrix}
      \cos \theta & -\sin\theta \\
      \sin\theta & \cos\theta
    \end{bmatrix}
  \]
  Eigenvalues
  \begin{align*}
    0&=\det\paren{R_\theta- t}\\
     &=
                              \begin{bmatrix}
                                \cos\theta - t & -\sin\theta \\
                                \sin\theta & \cos\theta - t
                              \end{bmatrix} \\
    0&=\paren{\cos\theta - t}^2 + \sin^2\theta \\
    \paren {\cos \theta - t}^2 &= -\sin^2\theta  \\
    \cos\theta - t &= \pm i\sin\theta \\
    t &= \cos\theta \pm i \sin\theta \\
    &= e^{i\theta}
  \end{align*}
  Eigenvectors
  \begin{align*}
    \begin{bmatrix}
      c & -s \\
      s & c
    \end{bmatrix}
    \begin{bmatrix}
      i \\ 1
    \end{bmatrix}
    &=
      \begin{bmatrix}
        ic & -s \\
        is & c
      \end{bmatrix}
      =
      \begin{bmatrix}
        i (c+is) \\
        1 (c+ is)
      \end{bmatrix} \\
    &= e^{i\theta}
      \begin{bmatrix}
        i \\ 1
      \end{bmatrix} \\
    \begin{bmatrix}
      c & -s \\
      s & c
    \end{bmatrix}
    \begin{bmatrix}
      1 \\ i
    \end{bmatrix}
    &=
      \begin{bmatrix}
        c -si \\
        s + ci
      \end{bmatrix}
      =
      \begin{bmatrix}
        1(c -si) \\
        i(c-si)
      \end{bmatrix} \\
    & =e^{-i\theta}
      \begin{bmatrix}
        1 \\ i
      \end{bmatrix}
  \end{align*}
  Diagonalized
  \begin{align*}
    \begin{bmatrix}
      c & -s \\ s & c
    \end{bmatrix}
    &=
    \begin{bmatrix}
      1 & i \\
      i & 1
    \end{bmatrix}
    \begin{bmatrix}
      e^{-i\theta} \\ & e^{i\theta}
    \end{bmatrix}
    \begin{bmatrix}
      1 & i \\
      i & 1
    \end{bmatrix}^* \\
    &=
    \begin{bmatrix}
      1 & i \\
      i & 1
    \end{bmatrix}
    \begin{bmatrix}
      e^{-i\theta} \\ & e^{i\theta}
    \end{bmatrix}
    \begin{bmatrix}
      1 & -i \\
      -i & 1
    \end{bmatrix}
  \end{align*}

  In general, by Jordan normal form,
  \begin{align*}
    R &\sim
    \begin{bmatrix}
      e^{\pm i \theta_0} \\ &e^{\pm i \theta_1} \\ &&\ddots
    \end{bmatrix} \\
    &\sim
    \begin{bmatrix}
      \begin{bmatrix}
        1 & i \\ i & 1
      \end{bmatrix}
      \begin{bmatrix}
        e^{-i\theta_0} \\ & e^{i\theta_0}
      \end{bmatrix}
      \begin{bmatrix}
        1 & -i \\ -i & 1
      \end{bmatrix}
      \\ & \ddots
    \end{bmatrix}
  \end{align*}
  I think I'm missing an idea to close this off
  \section{Rotor}
  \begin{align*}
    \R&=\b_0\b_1 \\
    \intertext{What does it do on \(\Span \b_0,\b_1\)?  The map \(\blank \R\) rotates counterclockwise (\(+\)):}
    \b_0\R &= \b_0\b_0\b_1 = \b_1  \\
    \b_1\R &= \b_1\b_0\b_1 = -\b_0 \\
    \intertext{And \(R\blank\) rotates clockwise (\(-\)) as \(\R\) anticommutes with \(\b_0,\b_1\):}
    \R\b_0 &= \b_0\b_1\b_0 = -\b_1 \\
    \R\b_1 &= \b_0\b_1\b_1 = \b_0
  \end{align*}
  Because \(\R\) commutes with the other \(\b_n\), this rotates the \(\b_0,\b_1\) plane and fixes the rest:
  \[
    e^{-t\R} \blank e^{t\R}:
    \begin{cases}
      \b_0 &\mapsto \b_0 e^{2t\R} \\
      \b_1 &\mapsto \b_1 e^{2t\R} \\
      \b_n &\mapsto \b_n
    \end{cases}
  \]

  Doran, Lasenby Geometric algebra for physicists 2.7 Rotations

  \begin{lemma}
    Unitary equivalence restricts to orthogonal equivalence
  \end{lemma}
  \begin{proof}
    ?
    % https://mathoverflow.net/questions/140878/can-two-unitary-similar-real-matrix-be-orthogonal-similar
    % https://math.stackexchange.com/questions/178145/a-unitary-matrix-taking-a-real-matrix-to-another-real-matrix-is-it-an-orthogona
  \end{proof}

\end{multicols*}
\end{document}
